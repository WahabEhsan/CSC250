\documentclass[12pt]{article}
\usepackage{mathtools}
\usepackage{amsthm}
\usepackage{enumerate}
\usepackage{amssymb}%
\usepackage{amsmath}%
\usepackage{pb-diagram}%
\usepackage{amscd}%
\usepackage{graphicx}
\usepackage{mathdesign}
\usepackage{bbding}

\usepackage[total={6in,9.7in},centering]{geometry}

\pagestyle{empty}

\setlength{\parindent}{0pt}


\newcommand{\ZZ}{\mathbb{Z}}
\newcommand{\QQ}{\mathbb{Q}}
\newcommand{\NN}{\mathbb{N}}
\newcommand{\RR}{\mathbb{R}}
\newcommand{\CC}{\mathbb{C}}
\newcommand{\MOD}{\mathrm{mod}}


\begin{document}

\begin{center}
\textsc{CSC 250 \qquad Assignment \#4 \qquad Wahab Ehsan}
\end{center}
\begin{enumerate}
\item (\S4.1 Problem 2b)\\
Write down all the properties that the following relation satisfies from among the properties reflexive, symmetric, transitive, irreflexive, and antisymmetric.
\[R=\{(a,b)\mid a^2=b^2\} \text{ over the real numbers.}\]
Answer : Reflexive and Symmetric 





%==============================================================================================

\item (\S4.1 Problem 12d)\\
Find the symmetric closure of the following relation over the set $\{a,b,c\}$.
\[R=\{(a,a),(a,b),(c,b),(c,a)\}\]
Answer : $s(R)=\{(a,a),(a,b),(b,a),(c,b),(b,c),(c,a),(a,c)\} $




%==============================================================================================







\item (\S4.1 Problem 13d)\\
Find the transitive closure of the following relation over the set $\{a,b,c,d\}$.
\[R=\{(a,b),(b,c),(c,d),(d,a)\}\]
Answer :\\ $s(R)=\{(a,b),(b,c),(a,c),(c,d),(d,a),(c,a),(a,a),(b,d),(a,d),(d,b),\\(c,b),(c,c),(d,d),(b,b),(b,a),(d,c)\} $




%==============================================================================================







\item (\S4.1 Problem 21)\\
Prove that if a relation $R$ has the symmetric property, then so does $R^2$.
\begin{proof}
Let $(a,b) \in R^2$ be arbitrary,
Since $(a,b)\in R^2$ there must be $(b,a)$ in $R$, but $R$ is symmetric, Thus $(a,b)\in R$. Since $(a,b),(b,a)\in R$, $(b,a)\in R^2$ 

\end{proof}





%==============================================================================================







\item (\S4.2 Problem 3d)\\
For the following function $f$ with domain $\NN$, describe the equivalence classes of the kernel of $f$.
\[f(x)=floor(x/3).\]
Answer : $[3n] = \{3n,3n+1,3n+2\}$ for each $n \in \NN$




%==============================================================================================








\item (\S4.2 Problem 6b)\\
Given the following set of words:
\[\{rot,\ tot,\ root,\ toot,\ roto,\ toto,\ too,\ to,\ otto\}\]
Let $f$ be the function that maps a word to its bag of letters. For the kernel relation of $f$, describe the equivalence classes. \\
\\
Answer : Six classes :\\
\[[t,o,o,t]=\{toot,toto,otto\}\]
\[[r,o,o,t]=\{root, roto\}\]
\[[r,o,t]=\{rot\}\]
\[[t,o,t]=\{tot\}\]
\[[t,o,o]=\{too\}\]
\[[t,o]=\{to\}\]





%==============================================================================================







\item (\S4.2 Problem 8)\\
Let $R$ be a relation on a set $S$ such that $R$ is symmetric and transitive and for each element $x\in S$ there is an element $y\in S$ such that $x\, R\, y$. Prove that $R$ is an equivalence relation (\emph{i.e}, prove that $R$ is reflexive).
\begin{proof}
For each $x\in S$, there exist $y \in S$ such that $(x,y) \in R$. Since $R$ is Symmetric and Transitive, there must be $(a,b),(b,a),(b,c),(a,c) \in R$, using $R^2$ and transitive property, there exist $(a,a) \in R^2$ which is the reflexive property. Hence $R$ is an equivalence relation.

\end{proof}





%==============================================================================================








\item (\S4.4 Problem 2b)\\
Use induction to prove the following equation:
\[5+9+11+\ldots+(2n+3)=n^2+4n.\]
\begin{proof} 
\hfill \break Basis : $n=1$ 
\[p(1)=(2(1)+3)=5\]
\[5=5, \ p(1)\text{ is true } \checkmark\]
Induction : \\
Assume $p(k)$ is true,
\[p(k+1)=5+9+11+\ldots+(2(k+1)+3).\]
\[=(k^2 + 4k) +(2(k+1)+3).\]
\[=k^2 + 6k + 5\]
\[=(k+1)(k+5)\]
\[=(k+1)((k+1)+4)\]
\[=(k+1)^2 + 4(k+1)\]
\[=p(k+1) \text{ is true.} \checkmark\]
\end{proof}





%==============================================================================================







\item (\S4.4 Problem 2h)\\
Use induction to prove the following equation:
\[2+6+12+\ldots+n(n+1)=\frac{n(n+1)(n+2)}{3}.\]
\begin{proof}
\hfill \break Basis : $n=1$ 
\[p(1)=(1)((1)+1)=2\]
\[2=2, \ p(1)\text{ is true } \checkmark\]
Induction : \\
Assume $p(k)$ is true,
\[p(k+1)=2+6+12+\ldots+(k+1)((k+1)+1)\]
\[=\frac{k(k+1)(k+2)}{3}+(k+1)((k+1)+1)\]
\[=\frac{k(k+1)(k+2)}{3}+\frac{3(k+1)((k+1)+1)}{3}\]
\[=\frac{k(k+1)(k+2)+3(k+1)((k+1)+1)}{3}\]
\[=\frac{k^3 + 6k^2 + 11k + 6}{3}\]
\[=\frac{(k+1)(k+2)(k+3)}{3}\]
\[=\frac{(k+1)((k+1)+1)((k+1)+2)}{3}\]
\[=p(k+1) \text{ is true.} \checkmark\]
\end{proof}





%==============================================================================================







\item (\S4.4 Problem 8)\\
Use induction to prove that a finite set with $n$ elements has $2^n$ subsets.
\begin{proof}
\hfill \break 
Let $P(n)$ be the statement "A finite set with $n$ elements has $2^n$ subsets." To prove this statement $P(n)$ has to be true for all $n \in \NN$. If the finite set has $0$ elements, then there will be $1$ subset. So the basis, $P(0)$ is true. Now assume that $P(k)$ is true for an arbitrary $k \geq 0$. Let $S$ be a finite set with $k+1$ elements. Let $power(S)$ be the set from $S$ which had the subsets removed at $k+1$ element. Then $power(S)$ is a finite set that has $k$ elements and, by assumption, at most $2^k$ subsets. It wasn't possible to remove more than $2$ subsets from $S$ for each of the $2^k$ subsets in $power(S)$. Therefore, $T$ must have $2 * 2^k = 2^{k+1}$ subsets. So, $P(k+1)$ is true. From the Principle of Mathematical Induction it is proven that $P(n)$ is true for all $n \in \NN$.


\end{proof}


\end{enumerate}
\end{document} 