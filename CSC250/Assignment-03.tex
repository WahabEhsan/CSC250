\documentclass[12pt]{article}
\usepackage{mathtools}
\usepackage{amsthm}
\usepackage{enumerate}
\usepackage{amssymb}%
\usepackage{amsmath}%
\usepackage{pb-diagram}%
\usepackage{amscd}%
\usepackage{graphicx}
\usepackage{mathdesign}

\usepackage[total={6in,9.7in},centering]{geometry}

\pagestyle{empty}

\setlength{\parindent}{0pt}

\newcommand{\ZZ}{\mathbb{Z}}
\newcommand{\QQ}{\mathbb{Q}}
\newcommand{\NN}{\mathbb{N}}
\newcommand{\RR}{\mathbb{R}}
\newcommand{\CC}{\mathbb{C}}
\newcommand{\MOD}{\mathrm{mod}}
\begin{document}

\begin{center}
\textsc{CSC 250 \\ Assignment \#3}\\
\textsc{Wahab Ehsan}
\end{center}





%============================================================================================


\begin{enumerate}
\item (Problem 1.b \S 3.1)\\
For the following inductive definition, start with the basis elements and construct 10 more elements in the set:\\

\textbf{Basis: }$1\in S$. \textbf{Induction: }If $x\in S$, $2x$, $2x+1\in S$.
 \\ Answer : $\{1, 2, 3, 4, 5, 6, 7, 8, 9, 10,11\}$








%============================================================================================


\item (Problem 2.b \S 3.1)\\
Find an inductive definition for the following set:

\[\{0,2,4,6,8,\ldots\}\]
Answer : 
\textbf{Basis: }$0\in S$. \textbf{Induction: }If $x\in S$, then $x+2 \in S$.








%============================================================================================


\item (Problem 6.d \S 3.1)\\
Find an inductive definition for the following set of strings:

\[\{a^mb^n\mid m,n\in\NN\}\]

Answer :
\textbf{Basis: }$a, b\in S$. \textbf{Induction: }If $x\in S$, then $axb\in S$.







%============================================================================================


\item (Problem 10.b \S 3.1)\\
Find an inductive definition for the following set of lists. Use the $cons$ constructor:

\[\{\langle1\rangle,\langle 2,1\rangle,\langle 3,2,1\rangle,\ldots\}\]

Answer :
\textbf{Basis: }$\langle 1 \rangle \in S$. \textbf{Induction: }If $L\in S$, then $ (head(L) + 1) :: L\in S$.


%============================================================================================


\item (Problem 2 \S 3.2)\\
Given the following definition for the length of a list:
\[length(L):= \text{ if } L=\langle\ \rangle \text{ then } 0 \text{ else } 1+length(tail(L)).\]

Write down each step in the evaluation of $length(\langle r,s,t,u\rangle)$.

\[length(\langle r,s,t,u\rangle)= 1+length(\langle s,t,u\rangle).\]
\[ \ \ \ \ \ \ \ \ \ \ \ \ \ \ \ \ \ \ \ \ \ \ \ \ \ \ =  1+1+length(\langle t,u\rangle).\]
\[ \ \ \ \ \ \ \ \ \ \ \ \ \ \ \ \ \ \ \ \ \ \ \ \ \ \ \ \ \ =  1+1+1+length(\langle u\rangle).\]
\[ \ \ \ \ \ \ \ \ \ \ \ \ \ \ \ \ \ \ \ \ \ \ \ \ \ \ \ \ \ \ \ \ \ =  1+1+1+1+length(\langle \rangle).\]
\[ \ \ \ \ \ \ \ \ \ \ \ \ \ \ \ \ \ \  \ \ \ =  1+1+1+1+0\]
\[ \  =  4\]
%============================================================================================


\item (Problem 4.b \S 3.2)\\
Construct a recursive definition for the following function, where all variables represent natural numbers:

\[f(n)=floor(0/2)+floor(1/2)+\ldots+floor(n/2)\]
Answer : 
\[f(0)=0 \text{ and } f(n)=f(floor((n-1)/2)) + floor(n/2)\]


%============================================================================================


\item (Problem 5.b \S 3.2)\\
Construct a recursive definition for the following string function of strings over the alphabet $\{a,b\}$:

\[f(x)=xy\text{, where }y\text{ is the reverse of }x.\]
Answer : 
\[f(\Lambda)=\Lambda \text{ and }f(ax) = af(x)a \text{ and } f(bx) = bf(x)b \]









\end{enumerate}
\end{document} 