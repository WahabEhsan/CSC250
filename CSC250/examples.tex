% CSC 250 LaTeX Examples

\documentclass[12pt]{article}

\usepackage{mathtools}
\usepackage{amsthm}
\usepackage{enumerate}
\usepackage{amssymb}%
\usepackage{amsmath}%
%\usepackage{pb-diagram}%
\usepackage{amscd}%
\usepackage{graphicx}
\usepackage{mathdesign}
\usepackage[total={6in,9.7in},centering]{geometry}

\pagestyle{empty}

\setlength{\parindent}{0pt}

%here we can declare our own commands
\newcommand{\ZZ}{\mathbb{Z}}

\begin{document}

\begin{center}
CSC 250 \qquad Examples
\end{center}

%using the enumeration package we can enumerate in any way we want
\begin{enumerate}[I.)]
\item Examples of numeration, some symbols and using our declared commands\\
%example of using the \newcommand at the top
$a\in \mathbb{Z}$\\
$a\in\ZZ$

%some math commands
$\leq$ less than or equal to\\
$<$ less than\\
$\geq$ greater than or equal to\\
$>$ greater than

\item next problem

\end{enumerate}


\begin{center}
Solutions
\end{center}
\begin{enumerate}[1.)]
\item (Problem 17.a \S1.1)\\
Prove the statement by using the indirect approach of proof by contradiction. In   the proof, come up with the contradiction that 1  is an even number.\\

``If $x$ and $y$ are even, then $x+y$ is even.''

\begin{proof}
Let $x$ and $y$ be even. Then $x=2k$ and $y=2h$ for some $h,\ k\in\ZZ$.\\
Now assume, to the contrary, that $x+y$ is odd, then $x+y=2z+1$ for some $z\in\ZZ$. So, we have
\[x+y=2z+1,\]
and by using substitution,
\[2k+2h=2z+1\]
\[2k+2h-2z=1\]
\[2(k+h-z)=1.\]
Thus, by this equality, 1 is an even number, which is never true. Hence, we have a contradiction.

Therefore, if $x$ and $y$ are even, then $x+y$ is even.


\end{proof}


%================================================================================





\newpage
\item (Problem 4.g \S1.1)\\
Write down a contrapositive version for each of the following conditional statements.

``If $x>0$ and $y>0$, then $x+y>0$.''

Answer:\\
``If $x+y\leq 0$, then $x\leq 0$ or $y\leq 0$.''\\

For the answer, we do no have to write this below, I am stating this here to establish a train of thought:\\

For the usual idea of ``If $A$, then $B$,'' we can take this apart and we have
\[A=``x>0 \text{ and } y>0''\]
and
\[B=``x+y>0.''\]
Lets take take the negation of these:\\
Since $A$ is a statement of two conditions, we use MeMorgan's Law here ($\neg(C\text{ and }D)=\neg C\text{ or } \neg D$)
\[\neg A=``x\leq 0 \text{ or } y\leq 0,''\]
and
\[\neg B=``x+y\leq 0.''\]
Now we simply use this to say  ``If $\neg B$, then $\neg A$.''


%================================================================================

\vskip 1in

\item (Problem 7.a \S1.1)\\
Prove or disprove the following statement by exhaustive checking.

``There is a prime number between $45$ and $54$.''

\begin{proof}$\ $\\
Starting our checking at $46$:\\
$46=2\times 23.$\\
$47$ is a prime (1 and 47 are its only divisors). 

Therefore the statement is true.

\end{proof}




\end{enumerate}


\end{document} 