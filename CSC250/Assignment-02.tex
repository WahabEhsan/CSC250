\documentclass[12pt]{article}
\usepackage{mathtools}
\usepackage{amsthm}
\usepackage{enumerate}
\usepackage{amssymb}%
\usepackage{amsmath}%
\usepackage{pb-diagram}%
\usepackage{amscd}%
\usepackage{graphicx}
\usepackage{mathdesign}

\usepackage[total={6in,9.7in},centering]{geometry}

\pagestyle{empty}

\setlength{\parindent}{0pt}

\newcommand{\ZZ}{\mathbb{Z}}
\newcommand{\QQ}{\mathbb{Q}}
\newcommand{\NN}{\mathbb{N}}
\newcommand{\RR}{\mathbb{R}}
\newcommand{\CC}{\mathbb{C}}
\newcommand{\MOD}{\mathrm{mod}}
\begin{document}

\begin{center}
\textsc{CSC 250 \\ Assignment \#2}\\
\textsc{Wahab Ehsan}
\end{center}





%============================================================================================


\begin{enumerate}
\item (Problem 5 \S 2.1)\\
Find an integer $n$ and a number $x$ such that $n\lfloor x\rfloor\not=\lfloor nx\rfloor$ and $n\lceil x\rceil\not=\lceil nx\rceil$. (show the results of each side of each equation as proof)

Let $x = 2.2$ and $n = 3.5$ in $n\lfloor x\rfloor\not=\lfloor nx\rfloor$ 

\[\ 6.5 \lfloor 4.2\rfloor = 26 \not= 27 = \lfloor 6.5 * 4.2\rfloor\]

Let $x = 2.2$ and $n = 3.5$ in $n\lceil x\rceil\not=\lceil nx\rceil$ 

\[\ 6.5 \lceil 4.2\rceil = 32.5 \not= 27 = \lceil 6.5 * 4.2\rceil\]


%============================================================================================


\vskip .4cm
\item (Problem 7 \S 2.1)\\
Evaluate the following expression using Euclid's algorithm (2.1.3). (\textbf{Write each step down for credit})
\[\gcd(98,\ 35)\]
\[\ 35 = 98 * 0 + 35\]
\[\ 98 = 35 * 2 + 28\]
\[\ 35 = 28 * 1 + 7\]
\[\ 28 = 7 * 4 + 0\]
\[\ 7\]



%============================================================================================


\vskip .4cm
\item (Problem 10 \S 2.1)\\
Evaluate the following expression.
\[-15\ \MOD\ 12\]
Answer : 9




%============================================================================================


\vskip .4cm
\item (Problem 13 \S 2.1)\\
Use the algorithm in example 9 to convert the decimal number 65 to a binary number.\\
\[65 = 2 * 32 + 1 \]
\[32 = 2 * 16 + 0 \]
\[16 = 2 * 8 + 0 \]
\[8 = 2 * 4 + 0 \]
\[4 = 2 * 2 + 0 \]
\[2 = 2 * 1 + 0 \]
\[1 = 2 * 0 + 1 \]
\[0. \ (done) \]
Answer : 1000001




%============================================================================================


\vskip .4cm
\item (Problem 18 \S 2.1)\\
Use the properties of $\log$ (2.1.5 page 97) to evaluate the following expression. 
Property used from (2.1.5f).
\[\log_{\sqrt{2}}2\]
\[{\sqrt{2}}^\log_{\sqrt{2}}2\] 
\[x = 2\]





%============================================================================================


\vskip .4cm
\item (Problem 31.d \S 2.1)\\
Prove that if $c$ is a common divisor of $a$ and $b$, then $c$ divides $\gcd(a,\ b)$. HINT: use properties of (2.1.2c) and (1.1.1).
\begin{proof}$\ $\\
Let $g = \gcd(a, b)$, so there are integers $x$ and $y$ such that $g = ax + by$ and assume $c$ is a common divisor of $a$ and $b$, then $c|a$ and $c|b$, so $c|g$. Therefore $c$ does divide  $\gcd(a,\ b)$.





\end{proof}






%============================================================================================


\vskip .4cm
\item (Problem 33.b \S 2.1)\\
Let $f: A\to B$ be a function, and let $E$ and $F$ be subsets of $A$. Prove the following fact:
\[f(E\cap F)\subseteq f(E)\cap f(F). \]
\begin{proof}$\ $\\
We'll prove both containment at once : $x \in f(E\cap F)$ iff $f(x) \in f(E)\cap f(F)$, where $y \in E \cap F$ iff $x = f(y)$, where $y \in E$ and $y \in F$ iff $x \in f(E) $ and $x \in f(F)$ iff $x \in f(E)\cap f(F)$ .


\end{proof}





%============================================================================================


\vskip .4cm
\item (Problem 3 \S 2.2)\\
Let $f(x)=x^2$ and $g(x,y)=x+y$. Find compositions that use the functions $f$ and $g$ for each of the following expressions.
\begin{enumerate}[a.]
\item $x^2+y^2$.
\\Answer : $g(f(x),f(y))$
\item $(x^2+y^2)^2+z^2$.
\\Answer : $g(f(g(f(x),f(y))), f(z))$

\end{enumerate}







%============================================================================================


\vskip .4cm
\item (Problem 4b \S 2.3)\\
Show that the following function $f:\NN\to\NN$ has the listed properties.
\begin{center}
$f(x)=x+1$\ \ \ \ \ \ \ \ \ \ \ \  injective and not surjective
\end{center}
Let $x_1, x_2 \in \NN$ and suppose $f(x_1) = f(x_2)$,
\[f(x_1) = f(x_2)\]
\[x_1 + 1 = x_2 + 1\]
\[x_1 = x_2\]
Thus showing $f(x)$ is injective.
Let $x \in \NN$ and let $x = y - 1$,
\[f(x) = x + 1\]
\[f(x) = (y - 1) + 1\]
\[f(x) = y\]
Thus showing $f(x)$ is surjective.





%============================================================================================


\vskip .4cm
\item (Problem 5.b \S 2.3)\\
Determine whether the following function is injective or surjective (or both or neither). 
\[f:\NN\to\NN\text{, where }f(x)=x\ \MOD\ 10.\]

Answer : $f(x)$ is surjective.





%============================================================================================


\vskip .4cm
\item (Problem 16 \S 2.3)\\
Let $f:A\to B$ and $g:B\to C$, prove:
\begin{center}
If $f$ and $g$ are surjective, then $g\circ f$ is surjective.
\end{center}
\begin{proof}$\ $\\
Assume $f$ and $g$ are surjective, let $c \in C$, then since $g$ and $f$ are subjective, there exist $b \in B$ and $a \in A$ such that $f(a) = b$ and $g(b) = c$. Therefore $g(f(a)) = g(b) = c$, which proves subjective.

\end{proof}







%============================================================================================


\vskip .4cm
\item (Problem 19b \S 2.3)\\
Let $g:A\to B$ and $h:A\to C$ and let $f:A\to B\times C$ be defined by $f(x)=(g(x),h(x))$. Show that the following statement holds.
\begin{center}
If $g$ or $h$ is injective, then $f$ is injective. After the proof, find an example to show that the converse is false.
\end{center}
\begin{proof}$\ $\\
Let $g$ or $h$ be injective, and let $x \in A$. So $f(g(x)) = f(h(x))$ hence proving injective. Now let $A = {1,2,3,4}$, $B = {5, 6, 7}$ and  $C = {8, 9, 10}$, The set $B X C$ has nine elements, and $A$ has four elements, so there cant be a injection.

\end{proof}



%============================================================================================



\vskip .4cm
\item (Problem 3 \S 2.4)\\
Use (2.4.3) to show that the following set is countable by describing the set as a union of countable sets.
\[\NN\times\NN\times\NN.\]
(TIP: This looks similar to Theorem 2.4.2 and has a similar idea, but you must use (2.4.3) to prove this.)
\begin{proof}$\ $\\
\\ Since $\NN \times \NN$ is a countable set, the sequence of natural numbers can have a union of countable set.
\end{proof}



%============================================================================================



\vskip .4cm
\item (Problem 6 \S 2.4)\\
Show that if $A$ is uncountable and $B$ is a countable subset of $A$, then the set $A-B$ is uncountable.
\begin{proof}$\ $\\
Assume $A$ is uncountable and $B$ is countable subset of $A$, since $B$ is a countable subset of $A$ and $B$ is uncountable, the difference of the countable and uncountable, or $|A|$ and $|B|$, set equal to uncountable, so $|A|-|B|$ is uncountable, therefore $A-B$ is uncountable.


\end{proof}
\end{enumerate}
\end{document} 